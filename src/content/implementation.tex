%----------------------------------------------------------------------------
\chapter{Implementation}
%----------------------------------------------------------------------------

%TODO: bevezetes
%----------------------------------------------------------------------------
\section{Tooling} \label{sec_tooling}
%----------------------------------------------------------------------------
%TODO: random bevezeto szoveg

%----------------------------------------------------------------------------
\subsection{Eclipse Environment} \label{subsec_emf}
%----------------------------------------------------------------------------
Eclipse is a popular, open-source integrated development environment (IDE). It is mainly used for Java-related application development, but also supports several other programming languages. It consists of a base workspace and an extensible plug-in system. Using this plug-in system, the develpment environment is easily customizable for different purposes, such as programming in different programming lanugages, modeling (using the Gamma Framework or Yakindu), or testing.

\textbf{Eclipse Modeling Framework}

The Eclipse Modeling Framework is an Eclipse-based modeling framework and code generation facility. It defines its own structured data model -- called Ecore -- for describing models and providing runtime support for the models. Models are defined using the XML Metadata Interchange (XMI) format, which is supported by various Eclipse plugins developed specifically for this purpose, as EMF is fully integrated into the Eclipse platform. It provides an environment to numerous technologies, including server solutions, persistence frameworks, UI and transformation frameworks.

%----------------------------------------------------------------------------
\subsection{Xtext Framework} \label{subsec_xtext}
%----------------------------------------------------------------------------
Xtext is an open-source framework for developing (mostly) domain-specific languages (DSLs). It has its own syntax for the definition of textual languages, resembling a context-free grammar extended with mappings to the in-memory representations. Unlike standard parser generators, it generates not only a parser, but also the abstract syntax tree (AST) of the grammar, and also support several other features, such as validation rules and editing support. This is because Xtext is based on the EMF project -- the metamodels of the defined languages are Ecore models --, and it is integrated into the Eclipse environment.

\textbf{Xtend}

Xtend is a general-purpose, high-level programming language based on Java. It is statically typed, object-oriented and uses the type system of Java. Xtend programs are compiled to Java code, thus allowing seamless integration with existing Java libraries. It provides numerous convenient extensions to Java, such as dispatch methods, type inference, operator overloading and extension methods.

%----------------------------------------------------------------------------
\subsection{Sirius} \label{subsec_sirius}
%----------------------------------------------------------------------------
Sirius is an open-source project for developing graphical modeling languages. It is integrated into the Eclipse environment, enabing the specification of viewpoints for EMF models, thus the creation of graphical views. In a Sirius workbench (editor), the elements of the viewpoint specification models are mapped to individual EMF model elements, thus allowing their graphical interpretation and editing. The whole viewpoint definition procedure is declarative, using OCL \cite{OCLStandard} (or Acceleo Query Language, AQL) expressions for the traversal of the diagram elements when needed.

Sirius supports various representation types. Traditional Sirius diagrams consist mainly of nodes and edges between nodes, suitable for models in which the position of the diagram elements carries no meaning - like several structural modeling languages. It also supports table and tree representations, and also \textit{sequence diagrams} for modeling behavior - in which the position on the diagram is also part of the semantics.

%----------------------------------------------------------------------------
\subsection{Owl} \label{subsec_owl}
%----------------------------------------------------------------------------

Owl \cite{Owl} is a tool collection for $\omega$-words, $\omega$-automata and linear temporal logic. It provides several algorithms for automata and LTL, supporting - among others - LTL expression parsing and simplification, reading and writing $\omega$-automata using the HOA format \cite{HOAFormat}, translation of LTL formula to $\omega$-automata with several possible acceptance conditions, and operations over $\omega$-automata, such as product, SCC decomposition emptiness checks and acceptance-condition transformations.

Through providing these algorithms, the library supports easy development and fast prototyping in the area of LTL and automata, thus also enabling rapid concept validation.

%----------------------------------------------------------------------------
\subsection{Automata Learning Framework} \label{subsec_automatonlearning}
%----------------------------------------------------------------------------
%TODO: korabbi automata learning framework

%----------------------------------------------------------------------------
\section{Interactive Learning Framework} \label{sec_interactivelearningframework}
%----------------------------------------------------------------------------
%TODO: random bevezeto szoveg

%----------------------------------------------------------------------------
\subsection{High-Level overview} \label{subsec_highleveloverview}
%----------------------------------------------------------------------------
%TODO: szep UML h hogyan terjesztettuk ki a keretrendszert, overview ezekrol

%----------------------------------------------------------------------------
\subsection{Oracle} \label{subsec_oracleimpl}
%----------------------------------------------------------------------------
%TODO: adaptive learnable + UI

%----------------------------------------------------------------------------
\subsection{Adaption layers and memoization} \label{subsec_memoization}
%----------------------------------------------------------------------------
%TODO: memoizinglearnable, teacher, etc

%----------------------------------------------------------------------------
\subsection{Adaptive Direct Hypothesis Construction} \label{subsec_adaptivedhc}
%----------------------------------------------------------------------------
%TODO: DHC

