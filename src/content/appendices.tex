%----------------------------------------------------------------------------
\appendix
%----------------------------------------------------------------------------
%\chapter*{\fuggelek}\addcontentsline{toc}{chapter}{\fuggelek}
%\setcounter{chapter}{\appendixnumber}
%\setcounter{equation}{0} % a fofejezet-szamlalo az angol ABC 6. betuje (F) lesz
%\numberwithin{equation}{section}
%\numberwithin{figure}{section}
%\numberwithin{lstlisting}{section}
%\numberwithin{tabular}{section}

%----------------------------------------------------------------------------
\chapter{LTL Expressions}
%----------------------------------------------------------------------------

%----------------------------------------------------------------------------
\section{The Syntax of the LTL Expressions}
%----------------------------------------------------------------------------
\begin{lstlisting} [language=tex,caption=Full syntax of the LTL expressions using the EBNF notation \cite{EBNFStandard},label=lst_ltlfullsyntax]
	LTLExpression = ArrowExpression;
	ArrowExpression = (OrExpression '->' ArrowExpression) |
	                    (OrExpression '<->' ArrowExpression) |
	                     OrExpression;
	OrExpression = (OrExpression '|' AndExpression) |
	                  AndExpression;
	AndExpression = (AndExpression '&' UntilExpression) |
	                   UntilExpression;
	UntilExpression = (FutureGloballyExpression 'U' UntilExpression) |
	                     FutureGloballyExpression;
	FutureGloballyExpression = ('F' NextExpression) |
	                             ('G' NextExpression) |
	                              NextExpression;
	NextExpression = ('X' PrimaryExpression) |
	                    PrimaryExpression;
	PrimaryExpression = ('(' LTLExpression ')') |
	                      ('!' PrimaryExpression) |
	                       LiteralExpression;
	LiteralExpression = AtomicProposition |
	                      'true' |
	                      'false';
	AtomicProposition = '^'?('a-z'|'A-Z'|'_') ('a-z'|'A-Z'|'_'|'.'|'0-9')*;		
\end{lstlisting}

%----------------------------------------------------------------------------
\clearpage\section{Válasz az ,,Élet, a világmindenség, meg minden'' kérdésére}
%----------------------------------------------------------------------------
A Pitagorasz-tételből levezetve
\begin{align}
c^2=a^2+b^2=42.
\end{align}
A Faraday-indukciós törvényből levezetve
\begin{align}
\rot E=-\frac{dB}{dt}\hspace{1cm}\longrightarrow \hspace{1cm}
U_i=\oint\limits_\mathbf{L}{\mathbf{E}\mathbf{dl}}=-\frac{d}{dt}\int\limits_A{\mathbf{B}\mathbf{da}}=42.
\end{align}
