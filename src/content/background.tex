%----------------------------------------------------------------------------
\chapter{Background}
%----------------------------------------------------------------------------
%TODO general chapter intro

%----------------------------------------------------------------------------
\section{Specifying Requirements} \label{sec_backgrspecreq}
%----------------------------------------------------------------------------

%----------------------------------------------------------------------------
\subsection{Requirements} \label{subs_backgrreq}
%----------------------------------------------------------------------------

Throughout this thesis, the concept of requirements is going to be used widely, therefore, it is essential to define it precisely. The definition given in \cite{sweterminology} is going to be used.

\begin{definition}[Requirement]
	\mbox{}
	\begin{enumerate}
		\setlength\itemsep{0.1em}
		\item A condition or capability needed by the user to solve a problem or achieve an objective.
		\item A condition or capability that must be met or possessed by a system component to satisfy a contract, standard, specification or other formally imposed documents.
		\item A documented representation of a condition or capability as in (1) or (2).
	\end{enumerate}
\end{definition}

Requirements are important, as the specification is present at both the beginning and the end of the software development project: the design can only start, if there are some requirements formulated, and acceptance tests are only possible in the presence of requirements.

Requirements can be specified in many different ways, the most common being textual requirements in traditional feature lists. This method is an informal way of requirements specification, as the structure of this format is hard to analyze due to it lacking a precise definition. Attempts were made to formalize this type of requirements by defining patterns and mapping the individual patterns to formal semantics, for instance in [TODO cite (form)], however, there are also less direct approaches, such as temporal logic. 

The rationale behind the formalization of requirements is the wide range of automatized applications called \textit{formal methods} -- such as validation, formal verification, test oracle generation and requirement documentation generation.

%----------------------------------------------------------------------------
\subsection{Linear-Time Temporal Logic} \label{subs_backgrltl}
%----------------------------------------------------------------------------



%----------------------------------------------------------------------------
\section{Model-Based Engineering} \label{sec_backgrmbe}
%----------------------------------------------------------------------------
%TODO explanation of the title

%----------------------------------------------------------------------------
\section{Formalisms for Modeling Behavior} \label{sec_backgrmodeling}
%----------------------------------------------------------------------------
%TODO DFA, NFA

%TODO LTS (Kripke?)

%TODO Mealy

%TODO Regular languages vs automata

%TODO Omega-regular languages if needed

%TODO Logics (esp LTL) -> ezt külön kiszedjük vagy ide be?

%----------------------------------------------------------------------------
\section{Automata Learning} \label{sec_backgrautomatalearning}
%----------------------------------------------------------------------------
%TODO active vs passive

%TODO Active

%TODO DHC